\documentclass{article}
\usepackage{inputenc}
\usepackage[T1]{fontenc}
\usepackage{graphicx}
\usepackage{multicol}
%\usepackage{glossary}
\usepackage{makeidx}
\usepackage{textcomp}
\usepackage{multirow}
\usepackage[francais]{minitoc}
\usepackage{tabularx}

%\usepackage{sidecap}
%\usepackage[small,bf]{caption}
%\usepackage{natbib}

%%%%%%%%%%%%%%%%%%%%%%%%%%%%%%%%%%%%%%%%%%%%%%%%%%%%%%%%%
% Si on crée des commandes, les ajouter ici. Exemple :

\newcommand{\mapletext}[1]{
\noindent \begin{tabular}{|p{0pt}p{5in}}
\cline{1-1}&#1\\
\cline{1-1}
\end{tabular}
\bigskip
}
\newtheorem{prop}{Propriété}
\newtheorem{guide}{Guideline}
%%%%%%%%%%%%%%%%%%%%%%%%%%%%%%%%%%%%%%%%%%%%%%%%%%%%%%%%%%
\begin{document}
%\dominitoc
%\faketableofcontents
%\tableofcontentsb
%\listoffigures
%\listoftables
\pagenumbering{arabic}

\label{resume}

\section*{R\'{e}sum\'{e}}



{\small Dans le domaine du g�nie logiciel pour les Interactions Homme Machine (IHM), la migration des interfaces utilisateurs (UI) est un moyen pour r\'{e}utiliser des applications sur des plateformes ayant des modalit\'{e}s d'interactions diff\'{e}rentes des environnements de d\'{e}part. Les approches existantes de migration des UI sont manuelles dans le cadre des approches sp\'{e}cifiques, elles sont automatiques dans le cadre des services d'adaptation des UI aux contextes d'usage, ou elles sont semi automatiques dans le cadre d'une migration flexible dirig\'{e}e par un concepteur.}

{\small Dans cette th�se nous nous int\'{e}ressons � la migration semi automatique des UI vers une cible comme une table interactive dans l'objectif de transformer des UI Desktop en UI qui favorisent la collaboration et l'utilisation des objets tangibles. Les tables interactives sont des plateformes qui disposent des instruments d'interactions permettant de d\'{e}crire des UI tangibles et multi-utilisateurs. En consid\'{e}rant que le noyau fonctionnel (NF) des applications de d\'{e}part peut �tre r\'{e}utilis\'{e}s sur les cibles sans changement, les UI des applications sont caract\'{e}ris\'{e}es par la dimension des dialogues entre les utilisateurs et le syst�me, la dimension de la structure et
du positionnement des \'{e}l\'{e}ments graphiques et la dimension du style des \'{e}l\'{e}ments visuels. La migration d'une UI dans ces conditions consiste � transformer ou � recr\'{e}er les diff\'{e}rentes dimensions d'une UI de d\'{e}part pour la cible tout en consid\'{e}rant les crit�res de conception des UI pour les tables interactives.}

{\small Nous proposons dans cette th�se un mod�le d'interactions abstraites pour \'{e}tablir les \'{e}quivalences entre les dialogues et la structure des UI ind\'{e}pendamment des modalit\'{e}s d'interactions des plateformes source et cible. Les primitives d'interactions et la structure des composants graphiques permettent de d\'{e}crire des op\'{e}rateurs d'\'{e}quivalences pour retrouver et classer les \'{e}l\'{e}ments graphiques \'{e}quivalents en prenant en compte les guidelines des tables interactives. Nous proposons aussi des r�gles de substitution et de concr\'{e}tisation pour accro�tre l'accessibilit\'{e} des \'{e}l\'{e}ments graphiques et favoriser l'utilisation des objets tangibles.
}

{\small \textbf{Mots cl\'{e}s :} migration des interfaces utilisateurs, \'{e}quivalences des plateformes, modalit\'{e}s d'interactions, crit�res de conception, guidelines}

\section*{Abstract}

{\small In software engineering, in the field of human computer interaction (HCI), the migration of user interface
(UI) is a way to reuse existing applications on platforms with different interactions modalities. The existing
approaches for UI migration can be manual (for specific applications), they can be automatic (for services
which adapt UI based on context aware), or they can be mix of the previous - semi automatic (providing a
flexible migration process driven by the person in charge).}


{\small This thesis proposes a semi automatic process for migration of UI from a desktop to interactive table for the
purpose of transforming the UI of desktop to support further collaboration and usage of tangible objects. The
interactive tables are platforms with interactions instruments which allow the describtion of tangible and multi
users UIs. Considering that the functional core (FC) of source applications can be reused on target platform
without transformation, any UI can be characterized with three dimensions : the first dimension concerns the
dialogues between the users and the system, the second dimension concerns the structure and the layout of
graphical components, and the third dimension concerns the visual style of graphical elements. In this context,
the problematic regarding the UI migration is how to transform or re inject these different dimensions of source
UI into the target, while considering the UI design criteria for interactive tables.}


{\small This thesis proposes an abstract interactions model for establishing equivalences (independent of modalities
of interactions) between the source and the dialogue and structure of the target. The primitives of interaction and
the structure of graphical components are used to describe equivalence operators to find and to rank equivalent
elements on interactive tables. Furthermore, this thesis proposes substitution and concretization rules to increase
the accessibility of graphical elements and to facilitate the usage of tangible objects. The ranking process and
the transformation rules are based on guidelines for UI migration to interactive tables which are interpreted
form design criteria.}

{\small \textbf{Keywords :} user interface migration, equivalence of plateform, design criteria, guidelines}
%\chapter*{Acknowledgements}
%\part{Domaine d'étude et état de l'art}
\chapter{Scénario de migration assistée vers une table interactive}
\label{chap1}
\section*{Contexte}
Le nombre grandissant de plateformes et surtout la tr�s grandes vari�t�s de dispositifs d'interactions dont ils disposent telles que les smartphones, les tablettes, les terminaux tactiles ou les tables
interactives ont g�n�ralis� l'usage des applications ayant des modalit�s d'interactions ~\cite{NC1993} beaucoup plus sophistiqu�es qu'une simple utilisation du clavier et de la souris. Les tables interactives par
exemple offrent la possibilit� de mettre en \oe{}uvre des applications r�ellement multi-utilisateurs m�lant
interactions tactiles et objets tangibles~\footnote{Les interfaces utilisateurs tangibles (TUI) permettent � une application de pouvoir interagir avec des objets physiques directement manipulable par les utilisateurs ~\cite{Ullmer1997}. Les objets tangibles sont ceux pouvant �tre manipul�s par une interface utilisateur tangible} sur une surface de pr�t de 1 m�tre carr�.

Le domaine du g�nie logiciel pour les Interactions Homme Machine (IHM) propose plusieurs
approches pour la conception d'une application. En particulier la tendance actuelle est de concevoir les Interfaces Utilisateurs (UI) selon les mod�les du Framework de R�f�rence Cameleon
(CRF) ~\cite{Calvary2002} qui identifie diff�rents niveau d'interaction. De la plus abstraite ou ind�pendante des dispositifs d'interactions disponibles sur la plateforme vers un niveau concret. Ceci permettant la mise en \oe{}uvre
de l'UI selon la modalit� d'interaction souhait�e et disponible sur la plateforme cible. Si l'interface utilisateur d'une application est construite selon cette approche, alors la migration de cette derni�re vers des terminaux ayant des modalit�s diff�rentes est abord� partiellement par ~\cite{Paterno2011}. Notre travail
aborde la possibilit� de faire migrer entre plateformes ayant des modalit�s d'interactions diff�rentes
des applications qui n'ont pas �t� con�ues selon les mod�les du framework de r�f�rence (CRF).

La migration des UI est une activit� de g�nie logiciel qui implique la transformation des diff�rents aspects qui constituent une UI existante tels que les interactions (ou le dialogue entre l'utilisateur et le syst�me) qu'il faut n�cessairement pr�server pour que l'utilisateur puisse toujours accomplir les m�mes t�ches, les structures (organisations et types de donn�es des �l�ments graphiques), le positionnement et les styles des composants graphiques qui doivent �tre adapt�s pour �tre conformes aux sp�cificit�s de la plateforme cible et toujours satisfaire les utilisateurs finaux dans les choix de configuration qui ont put �tre fait sur la source.

Au del� des probl�mes li�s � des diff�rences possibles entre les environnements d'ex�cution des
plateformes source et cible  ~\cite{TKB78}qui n�cessite le portage du code, les diff�rences des modalit�s d'interactions entre la source et la cible impliquent n�cessairement la prise en compte des crit�res usuels de conception des UI. La transformation de l'UI de d�part doit �tre guid�e non seulement par les crit�res ergonomiques de conception mais aussi par les dispositifs d'interactions disponible sur la plateforme cible qu'il peut �tre int�ressant d'utiliser.

�videmment, en compl�ment de la volont� de rendre disponible des dispositifs d'interactions nouveaux, peut-�tre plus intuitifs, les crit�res ergonomiques de conception constituent un ensemble de principes � respecter pendant la mise en \oe{}uvre. Ils permettent de garantir l'utilisabilit� d'une UI. Par exemple, l'utilisation d'une application pour desktop sur une table interactive sans aucune adaptation pose des probl�mes d'utilisabilit�, car la simulation du clavier et de la souris n'est pas le meilleur mode d'interaction en terme d'utilisabilit�. Si l'application le permet, on peut m�me imaginer une
utilisation multi-utilisateurs avec une nouvelle UI � d�duire de l'UI d'origine.


\section*{Enjeux de la migration}
Dans le cadre de la migration des applications pour desktops vers les tables interactives par exemple, nous notons que les dialogues, la structure et le positionnement des �l�ments des UI doivent prendre en compte l'�ventualit� d'avoir plusieurs utilisateurs mais surtout la possibilit� d'utiliser des objets tangibles. Travailler sur ces deux exemples va nous permettre d'avoir une r�elle �volutivit� des UI entre le terminal source et le terminal cible et permet de favoriser la collaboration dans un espace de travail co-localis� et multi-utilisateurs offert par la table interactive et donc de l'utiliser pleinement.

Nous faisons l'hypoth�se que le c\oe{}ur de l'application peut �tre migr� sans modification et qu'il ne faut agir qu'au niveau de l'UI qu'il faut bien �videmment faire migrer. Cette migration, d'un terminal
source vers un terminal cible donn�e doit permettre de conserver dans la mesure du possible les
dialogues, la structure et les positionnements relatifs des �l�ments de l'interface, le respect du style
des �l�ments graphiques des UI de la source mais bien �videmment proposer une adaptation pour utiliser au mieux, sans perdre les utilisateurs, les nouveaux moyens d'interaction mis � la disposition des utilisateurs.

La transformation des dialogues d'une UI desktop vers une cible collaborative peut-elle garantir
� chaque utilisateur des interactions qui favorisent la collaboration? De nombreux �l�ments sont �
prendre en compte. En effet, chaque dialogue ou message d'erreur ne doit pas perturber les activit�s
des autres utilisateurs; par exemple les ?feedbacks? ou les messages d'erreurs destin�s � un utilisateur
ne doivent pas bloquer un autre utilisateur si une r�ponse est attendue. Par ailleurs la transformation
des dialogues peut-elle assurer que chaque dialogue reste coh�rent avec l'intention et les actions des
utilisateurs ? Par exemple la modification d'une valeur par deux utilisateurs distincts peut les perturber.
Il faut en effet d�cider si le r�sultat est la somme des deux modifications, la moyenne, le min ou le
max. �videmment le contexte et la nature de la valeur peuvent guider sur le choix � effectuer.

La transformation des dialogues sur les UI et les fonctionnalit�s de l'application source peuvent
provoquer de nombreux changements? En effet, la transformation des dialogues des UI desktops
pour les tables interactives peuvent par exemple impliquer des modifications pour prendre en compte
la pr�sence de plusieurs utilisateurs ou d'objets tangibles.

La transformation de la structure et du positionnement des �l�ments d'une UI pour desktop vers
une table interactive consiste � assurer l'accessibilit� aux diff�rents utilisateurs des �l�ments graphiques pertinents~\footnote{Un menu par exemple}. Par ailleurs la transformation des aspects structurels d'une UI doit elle aussi pr�server au mieux l'utilisabilit� de l'UI. En effet les solutions de transformations propos�es doivent
produire des UI conformes aux sp�cificit�s de la plateforme cible et satisfaisantes pour les utilisateurs
finaux.

Pour terminer, la transformation du style a n�cessairement un impact sur la collaboration ou sur
l'utilisation des objets tangibles. G�n�ralement, chaque application adopte une charte graphique qu'il
faut soit respecter, soit r�nover.

Les questions soulev�es par les transformations des diff�rents aspects des UI lors d'une migration
sont usuellement trait�es de diff�rentes mani�res. En premier lieu, il est possible d'adopter une approche manuelle ~\cite{WGM08}. Celle-ci permet d'avoir, au prix d'un travail minutieux, une UI conforme
aux attentes des utilisateurs finaux car les transformations sont flexibles et donc la qualit� de l'UI
produite d�pend directement des comp�tences de celui qui effectue la migration manuelle. Mais cette approche est difficilement r�utilisable, � moins de la consigner dans un document, que ce soit pour
d'autres applications ou pour d'autres personnes car elle n�cessite une bonne connaissance des crit�res de transformation des diff�rents aspects et des technologies des UI des plateformes source et
cible.

Il est aussi possible d'adopter une approche automatique bas�e ou non sur des mod�les abstraits ~\cite{Besacier2010}~\cite{Paterno2010}  qui permet une transformation des diff�rents aspects des UI. Bien que ces approches automatiques soient r�utilisables, elles sont n�cessairement moins flexibles car le concepteur
qui ne peut pas intervenir pendant la transformation, adapte l'UI produite dans un second temps.


Entre ces deux approches, il est aussi possible d'adopter une approche semi automatique~\cite{Markopoulos1997}
qui pr�sente de nombreux inconv�nients car non seulement elle induit un travail suppl�mentaire pour
la personne en charge de la migration qui guide la transformation mais en permettant plus de flexibilit�,
il est assez difficile de capitaliser sur le travail effectuer si la personne en charge de la migration
change. N�anmoins, cette approche permet une transformation interactive des diff�rents aspects de
l'UI. L'int�r�t d'une flexibilit� dans l'approche de migration d'UI permet d'avoir des UI migr�es
proches des attentes des utilisateurs finaux.

\section*{Contribution de la th�se}
Cette th�se a pour premier objet d'�tudier la migration des UI entre plateformes ayant des dispositifs d'interactions diff�rents. Nous avons fait le choix de cibler en particulier la migration des
UI depuis une station poss�dant clavier et souris vers une table interactive. Nous souhaitons que
cette migration se fasse au moindre co�t pour les concepteurs ou les d�veloppeurs tout en prenant en
compte les sp�cificit�s de la cible et, bien �videmment, les crit�res ergonomiques usuels utilis�s pour
la conception des UI.

La solution que nous proposons repose sur un processus semi automatique de migration d'UI.
Celui-ci comporte plusieurs �tapes interactives pour que les les concepteurs ou les d�veloppeurs
puissent effectuer des choix conformes aux crit�res de conception qu'ils souhaitent privil�gi�s. Nous
avons fait le choix de l'interactivit�, bas� sur des choix simples pour garantir simultan�ment la r�utilisabilit�, minimiser les co�ts mais aussi pour accro�tre la flexibilit� et garantir ainsi l'UI la plus
pertinente.

Il est primordial de prendre en compte, lors de la migration des interfaces utilisateurs, les crit�res ergonomiques de conception. Nos travaux utilisent des concepts issus de plusieurs domaines de
recherche.

Dans le domaine de l'utilisabilit�, nous nous sommes int�ress�s aux travaux qui mod�lisent les
crit�res ergonomiques de conception pour les traduire en r�gles op�rationnelles et utilisables pendant
la conception. L'objectif est de pouvoir int�grer ces r�gles dans la plateforme de migration dans le but
de r�duire la charge de travail des personnes en charge de la migration.

Dans le domaine de l'ing�nierie des mod�les, nous nous sommes int�ress�s aux travaux permettant
de mod�liser une plateforme dans le but d'effectuer la migration � un niveau abstrait : de concept
� concept. L'objectif est de rendre notre travail r�utilisable si la source et la cible �voluent. Cette
approche nous permet d'abstraire non seulement les UI mais aussi les interactions. Nous proposons
ainsi un mod�le d'interactions bas�es sur les primitives d'interactions~\cite{Kalawa2011} pour d�crire les
actions atomiques qui constituent les dialogues entre l'utilisateur et le syst�me.


\section*{Plan du manuscrit}

Notre manuscrit est structur� de la mani�re suivante :

Le chapitre~\ref{chap2} pr�sente l'espace des probl�mes li�s � la transformations des diff�rents aspects d'une
UI. Il d�limite le p�rim�tre de la migration des UI d'un poste fixe vers une table interactive. Il fixe nos
objectifs.

Le chapitre~\ref{chap3} pr�sente une �tude du mod�le d'interactions des tables interactives dans le but
d'identifier les sp�cificit�s et les crit�res ergonomiques de conception � int�grer pour la migration
des UI vers cette cible.

Le chapitre~\ref{chap4} est un �tat de l'art des approches de migration des UI. Dans ce chapitre nous d�crivons les crit�res n�cessaires pour atteindre nos objectifs et nous �valuons les diff�rentes approches �
l'aide de ces crit�res. Ce chapitre se termine par une synth�se des diff�rentes approches pr�sent�es.


Le chapitre~\ref{chap5} propose un mod�le d'UI qui prend en compte deux aspects des UI : leurs interactions
et leur structure. Les objectifs de ce mod�le d'UI sont de d�crire les UI � migrer ind�pendamment des
plateformes mais aussi de d�crire des op�rateurs d'�quivalences entre les dispositifs d'interactions des
plateformes source et cible.

Le chapitre~\ref{chap6} r�sente les m�canismes de transformation de la structure et des interactions de l'UI
source vers la cible. L'objectif de ce chapitre est de d�crire la prise en compte effective des crit�res
ergonomiques de conception sous forme de guidelines par les m�canismes de transformation.

Le chapitre~\ref{chap7} pr�sente le prototype que nous avons r�alis�. Il constitue une preuve de concept des
m�canismes propos�s. Plusieurs applications ont �t� migr�es afin de valider les �l�ments des diff�rents
mod�les, mettre en �vidence le respect des crit�res ergonomiques et surtout mettre en �vidence les
b�n�fices que l'on peut retirer d'une telle approche.

Le chapitre~\ref{chap8} est une conclusion. Elle rappelle les objectifs que nous souhaitions atteindre, met
en �vidence nos contributions et leurs apports. Elle donne aussi quelques �l�ments sur des travaux
compl�mentaires qui pourraient �tre men�s pour parfaire nos travaux.



\bibliographystyle{alpha}
\bibliography{bib/biblio}

\end{document}
