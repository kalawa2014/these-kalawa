
\section*{R\'{e}sum\'{e}}



{\small Dans le domaine du g�nie logiciel pour les Interactions Homme Machine (IHM), la migration des interfaces utilisateurs (UI) est un moyen pour r\'{e}utiliser des applications sur des plateformes ayant des modalit\'{e}s d'interactions diff\'{e}rentes des environnements de d\'{e}part. Les approches existantes de migration des UI sont manuelles dans le cadre des approches sp\'{e}cifiques, elles sont automatiques dans le cadre des services d'adaptation des UI aux contextes d'usage, ou elles sont semi automatiques dans le cadre d'une migration flexible dirig\'{e}e par un concepteur.}

{\small Dans cette th�se nous nous int\'{e}ressons � la migration semi automatique des UI vers une cible comme une table interactive dans l'objectif de transformer des UI Desktop en UI qui favorisent la collaboration et l'utilisation des objets tangibles. Les tables interactives sont des plateformes qui disposent des instruments d'interactions permettant de d\'{e}crire des UI tangibles et multi-utilisateurs. En consid\'{e}rant que le noyau fonctionnel (NF) des applications de d\'{e}part peut �tre r\'{e}utilis\'{e}s sur les cibles sans changement, les UI des applications sont caract\'{e}ris\'{e}es par la dimension des dialogues entre les utilisateurs et le syst�me, la dimension de la structure et
du positionnement des \'{e}l\'{e}ments graphiques et la dimension du style des \'{e}l\'{e}ments visuels. La migration d'une UI dans ces conditions consiste � transformer ou � recr\'{e}er les diff\'{e}rentes dimensions d'une UI de d\'{e}part pour la cible tout en consid\'{e}rant les crit�res de conception des UI pour les tables interactives.}

{\small Nous proposons dans cette th�se un mod�le d'interactions abstraites pour \'{e}tablir les \'{e}quivalences entre les dialogues et la structure des UI ind\'{e}pendamment des modalit\'{e}s d'interactions des plateformes source et cible. Les primitives d'interactions et la structure des composants graphiques permettent de d\'{e}crire des op\'{e}rateurs d'\'{e}quivalences pour retrouver et classer les \'{e}l\'{e}ments graphiques \'{e}quivalents en prenant en compte les guidelines des tables interactives. Nous proposons aussi des r�gles de substitution et de concr\'{e}tisation pour accro�tre l'accessibilit\'{e} des \'{e}l\'{e}ments graphiques et favoriser l'utilisation des objets tangibles.
}

{\small \textbf{Mots cl\'{e}s :} migration des interfaces utilisateurs, \'{e}quivalences des plateformes, modalit\'{e}s d'interactions, crit�res de conception, guidelines}

\section*{Abstract}

{\small In software engineering, in the field of human computer interaction (HCI), the migration of user interface
(UI) is a way to reuse existing applications on platforms with different interactions modalities. The existing
approaches for UI migration can be manual (for specific applications), they can be automatic (for services
which adapt UI based on context aware), or they can be mix of the previous - semi automatic (providing a
flexible migration process driven by the person in charge).}


{\small This thesis proposes a semi automatic process for migration of UI from a desktop to interactive table for the
purpose of transforming the UI of desktop to support further collaboration and usage of tangible objects. The
interactive tables are platforms with interactions instruments which allow the describtion of tangible and multi
users UIs. Considering that the functional core (FC) of source applications can be reused on target platform
without transformation, any UI can be characterized with three dimensions : the first dimension concerns the
dialogues between the users and the system, the second dimension concerns the structure and the layout of
graphical components, and the third dimension concerns the visual style of graphical elements. In this context,
the problematic regarding the UI migration is how to transform or re inject these different dimensions of source
UI into the target, while considering the UI design criteria for interactive tables.}


{\small This thesis proposes an abstract interactions model for establishing equivalences (independent of modalities
of interactions) between the source and the dialogue and structure of the target. The primitives of interaction and
the structure of graphical components are used to describe equivalence operators to find and to rank equivalent
elements on interactive tables. Furthermore, this thesis proposes substitution and concretization rules to increase
the accessibility of graphical elements and to facilitate the usage of tangible objects. The ranking process and
the transformation rules are based on guidelines for UI migration to interactive tables which are interpreted
form design criteria.}

{\small \textbf{Keywords :} user interface migration, equivalence of plateform, design criteria, guidelines}